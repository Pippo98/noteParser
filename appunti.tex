\documentclass{article}%
\usepackage[T1]{fontenc}%
\usepackage[utf8]{inputenc}%
\usepackage{lmodern}%
\usepackage{textcomp}%
\usepackage{lastpage}%
\usepackage[tmargin=1cm,lmargin=1cm]{geometry}%
\usepackage{amsmath}%
%
%
%
\begin{document}%
\normalsize%
\section{ Diffusione }%
\label{sec:Diffusione}%
E' un processo che avviene in funzione della temperatura.\newline%
%
Difetti reticolari sono alla base della diffusione, in particolare le vacanze\newline%
%
Sfruttato nella saldatura\newline%
%
Avviene tra due materiali adiacenti, nel/nei punti di contatto avviene a diffusione, ovvero parte del materiale A\newline%
%
è andato ad occupare spazio nel materiale B e viceversa , è un processo che richiede molto tempo.\newline%
%
Dipende dalle dimesioni degli atomi dei due materiali, infatti il materiale con dimensioni più piccole è più \newline%
%
propenso a diffondere nel materiale con atomi più grandi.\newline%
%
La vibrazione degli atomi all'interno del solido provoca uno spostamento di essi, crea quindi vacanze e permette\newline%
%
a queste vacanze di potersi muovere, il processo di diffusione sfrutta questo principio, il risultato è quindi uno\newline%
%
spostamento di atomi da A a B attraverso questi "vuoti" all'interno del solido.\newline%
%
La temperatura va ad agire sulla vibrazione degli atomi e quindi sulla velocità di spostamento di essi, quindi\newline%
%
anche alla velocità del processo di diffusione.\newline%
%
\newline%
%
\newline%
%
Diagrammi che descrivono lo stato di un materiale date delle condizioni di contorno.\newline%
%
Analizzeremo sistemi binari, diagrammi binari, in funzione della variabile temperatura.\newline%
%
Confrontiamo la stabilità termodinamica di diverse fasi.\newline%
%
E' un'analisi indipendente dal tempo\newline%
%
\newline%
%
\subsection{ Modello }%
\label{subsec:Modello}%
Il primo modello e più semplice da analizzare è quello della diffusione stazionaria attraverso una lamina sottile.\newline%
%
La concentrazione delle die facce della lamina è costante nel tempo.\newline%
%
Prima legge di Fick descrive questo modello:\newline%
%
\begin{alignat}{2}%
J = -D * dC/dx
%
\end{alignat}%
Ovvero J è il flusso, (quantità di massa per quantità di area per unità di tempo) e dipende da:\newline%
%
D coefficente di diffusione,\newline%
%
C concentrazione\newline%
%
x posizione\newline%
%
il segno "{-}" è perchè il gradiente di concentrazione è negativo, per avere un flusso positivo serve quindi il segno "{-}"\newline%
%
\newline%
%
Se il gradiente di concentrazione non è costante nel tempo la prima legge non è più utilizzabile e viene sostituita dalla\newline%
%
Seconda legge di Fick\newline%
%
E' un'equazione che dipende dal tempo:\newline%
%
\begin{alignat}{2}%
dC/dt = D dC^2/dx^2
%
\end{alignat}%
\newline%
%
\textbf{ Esempio \newline%
}%
\textit{Il profilo di concentrazione è un grafico che per ascisse ha la posizione e per ordinate ha la concentrazione\newline%
}%
\textit{Un profilo di concentrazione varia nel tempo, infatti la tempo t1 sarà un'iperbole molto schiacciata verso l'origine,\newline%
}%
\textit{man mano che il tempo passa, l'iperbole si allarga, la concentrazione di materiale diffuso aumenta.\newline%
}%
\textit{n.b. il la curva non è un'iperbole\newline%
}%
\newline%
%
La seconda equazione di fick non è risolvibile analiticamente, può essere nel caso specifico.\newline%
%
Ad esempio nel caso della diffusione cel carbonio nel ferro viene risolta come:\newline%
%
\newline%
%
\begin{alignat}{2}%
(Cx - C0) / (Cs - C0) = 1 - erf(x / 2*sqrt(tD))
%
\end{alignat}%
\newline%
%
Cx: concenrazione in x\newline%
%
C0: concentrazione corrispondente alla concentrazione iniziale e a distanza infinita\newline%
%
Cs: concentrazione in superficie\newline%
%
erf: "l'erfiano" funzione matematica\newline%
%
\newline%

%
\subsection{ Erfiano }%
\label{subsec:Erfiano}%
Funzione di errore gaussiana\newline%
%
\newline%

%
\subsection{ Coefficente di diffusione }%
\label{subsec:Coefficentedidiffusione}%
E' quel parametro che descrive il comportamento della diffusione\newline%
%
Dipende dal materiale che diffonde e dal materiale in cui si diffonde\newline%
%
Dipende dalla temperatura\newline%
%
L'equazioione è:\newline%
%
\begin{alignat}{2}%
D = D0 - exp(- Q/(RT))
%
\end{alignat}%
\newline%
%
Q: energia di attivazione del processo diffusivo, dipende dai due materiali\newline%
%
R: costante dei gas\newline%
%
T: temperatura in Kelvin\newline%
%
\newline%

%
\subsection{ Vacanze Vs Interstiziali }%
\label{subsec:VacanzeVsInterstiziali}%
Il processo di diffuzione tramite vacanze sfrutta lo spazio creato dalle vacanze per accogliere atomi sostituiti.\newline%
%
Il processo di diffusione tramite interstizi avviene quando gli atomi dei due materiali hanno dimensioni molto diverse,\newline%
%
di conseguenza l'atomo con dimensione minore diffonde in quello di dimensione maggiore andando ad occupare gli spazi\newline%
%
presenti nel reticolo cristallino.\newline%
%
\newline%
%
Il coefficente di diffusione varia in base alla temperatura, nel caso di diffusione tramite vacanze, il coefficente di diffusione varia\newline%
%
di molto rispetto ad una diffusione tramite interstizi. Questo avviene perché il numero di vacanze e ls velocità di spostamento \newline%
%
delle vacanze dipende dalla temperatura, invece il numero o la dimensione degli interstizi rimane più o meno invariato.\newline%
%
\newline%

%
\subsection{ Drive In }%
\label{subsec:DriveIn}%
Pre{-}deposizione: fase iniziale in cui aspettiamo che si diffonda la qunatità di materiale desiderata\newline%
%
Fase nel processo di diffusione secondaria alla fase pre{-}deposizione in cui il profilo di concetrazione si appiattisce,\newline%
%
significa che il materiale diffuso si espande verso l'interno del materiale.\newline%
%
In questa fase la qunatità di materiale diffuso non aumenta, cambia invece la profondità a cui lo troviamo.\newline%
%
In generale in questa fase viene aumentata ulteriormente la temperatura per velocizzare il processo.\newline%
%
\newline%

%
\subsection{ Semiconduttori }%
\label{subsec:Semiconduttori}%
Semiconduttore è silicio o silicio{-}germaio, è necessario introdurre delle impurità che sono esssenziali per raggiungere\newline%
%
le porprietà elettriche desiderate.\newline%
%
\newline%
%
Ad esempio la diffusione può avvenire tra metallo e gas, in cui il sistema viene portato ad una temperatura di 900{-}1000 K.\newline%
%
\newline%
%
Per creare interconnessioni neli microprocessori viene tenuto conto anche del processo diffusivo.\newline%
%
Le interconnessioni sono i collegamenti tra componenti in silicio drogato, servono solo per trasportare corrente, +non agiscono attivamnete.\newline%
%
La scelta del materiale che viene utilizzato per un'interconnessione dipende dal fatto che durante il processo di\newline%
%
creazione di un microprocessore nella fase di creazione di interconnessioni, la temperatura deve rimanere inferiore ai 500°C.\newline%
%
A questa temperatura i metalli diffondono, l'esigenza è che il silicio rimanga puro/impuro come prima di iniziare la creazione di interconnessioni,\newline%
%
verrà quindi scelto il metallo con coefficente di diffusione più basso.\newline%
%
\newline%
%
Dati a 500 °C\newline%
%
Cu in Si {-}> 4   * 10\^{}{-}13\newline%
%
Au in Si {-}> 2.5 * 10\^{}{-}15\newline%
%
Ag in Si {-}> 4.2 * 10\^{}{-}17\newline%
%
Al in Si {-}> 2.5 * 10\^{}{-}21\newline%
%
\newline%
%
Diventa evidente che l'alluminio diventa il materiale scelto, nonostante non sia un buon conduttote come gli altri.\newline%
%
\newline%

%
\section{ Diagrammi di Stato }%
\label{sec:DiagrammidiStato}%
E' un processo che avviene in funzione della temperatura.\newline%
%
Difetti reticolari sono alla base della diffusione, in particolare le vacanze\newline%
%
Sfruttato nella saldatura\newline%
%
Avviene tra due materiali adiacenti, nel/nei punti di contatto avviene a diffusione, ovvero parte del materiale A\newline%
%
è andato ad occupare spazio nel materiale B e viceversa , è un processo che richiede molto tempo.\newline%
%
Dipende dalle dimesioni degli atomi dei due materiali, infatti il materiale con dimensioni più piccole è più \newline%
%
propenso a diffondere nel materiale con atomi più grandi.\newline%
%
La vibrazione degli atomi all'interno del solido provoca uno spostamento di essi, crea quindi vacanze e permette\newline%
%
a queste vacanze di potersi muovere, il processo di diffusione sfrutta questo principio, il risultato è quindi uno\newline%
%
spostamento di atomi da A a B attraverso questi "vuoti" all'interno del solido.\newline%
%
La temperatura va ad agire sulla vibrazione degli atomi e quindi sulla velocità di spostamento di essi, quindi\newline%
%
anche alla velocità del processo di diffusione.\newline%
%
\newline%
%
\newline%
%
Diagrammi che descrivono lo stato di un materiale date delle condizioni di contorno.\newline%
%
Analizzeremo sistemi binari, diagrammi binari, in funzione della variabile temperatura.\newline%
%
Confrontiamo la stabilità termodinamica di diverse fasi.\newline%
%
E' un'analisi indipendente dal tempo\newline%
%
\newline%
%
\subsection{ Fase }%
\label{subsec:Fase}%
Una fase è una porzione di materiale omogenea dal punto di vista chimico e fisico anche a livello microscopico.\newline%
%
Sono fasi diverse due materiali solidi uguali con struttura cristallina diversa.\newline%
%
Ad esempio Diamante e Grafite, stessa composizione ma struttura diversa.\newline%
%
Sono fasi diverse Acqua e Ghiaccio.\newline%
%
Dono fasi diverse Acqua e Olio.\newline%
%
Sono Fasi uguali Acqua e Alcool, infatti è una soluzione omogenea.\newline%
%
\newline%
%
I metalli allo stato liquido sono sempre perfettamente solubili, qualunque sia la combinazione.\newline%
%
\newline%
%
l'interfase è uno stato in cui coesistono più fasi contemporaneamente,\newline%
%
corrisponde alla linea di separazione tra le due fasi nel diagramma di stato\newline%
%
\newline%
%
\textbf{ Esempio \newline%
}%
\textit{Diagramma di stato di conposto binario acqua{-}zucchero.\newline%
}%
\textit{E' una fase quando la soluzione non è satura, appena lo diventa, allora sono distinte le due fasi, solida e liquida.\newline%
}%
\textit{Il valore di saturazione dipende dalla temperatura, ecco che viene introdotta la variabile che ci permette di descrivere un andamento.\newline%
}%
\newline%

%
\subsection{ Rame{-}Nichel }%
\label{subsec:Rame{-}Nichel}%
Diagramma di stato, ha come variabili temperatura e concentrazione di Nichel.\newline%
%
Ascisse: Concentrazoine percentuale.\newline%
%
Ordinate: Temperatura °C 1000{-}1600\newline%
%
Retta che parte da T: 1085 e \%=0 e arriva a T: 1435 e \%=100. Viene definita come: Curva di Solidus.\newline%
%
Curva sopra alla retta che parte da T: 1085 e \%=0 e arriva a T: 1435 e \%=100. Viene definita come: Curva di Liquidus.\newline%
%
\newline%
%
Al di sopra della curva vi è la fase liquida.\newline%
%
Al di sotto della retta vi è la fase solida.\newline%
%
\newline%
%
Tra la curva e la retta (simile ad una lente) vi è la coesistenza tra le due fasi solido e liquido della lega.\newline%
%
\newline%
%
Per qualsiasi punto posizionato tra retta e curva possiamo determinare le concentrazioni delle due fasi.\newline%
%
Per calcolare questo punto si comincia disegnando una linea isoterma, quindi una linea orizzontale, che va quindi ad intersecare\newline%
%
la curva di liquidus e la curva di solidus. I punti di intersezione corrispondono alla composizione della fase liquida e alla\newline%
%
composizione della fase solida. Questa non è ancora una quatità di fasi, definisce solo la composizione delle due fasi.\newline%
%
Per arrivare alla quantità delle due fasi è necessario applicare una formula che mette in relazione le concentrazioni.\newline%
%
Viene sfruttato il fatto che la concentrazione totale delle due fasi non cambia, quindi se a 1000 °C (ovvero monofasico solido) la\newline%
%
concentrazione era 65{-}35, anche a 1250°C la concentrazione dovrà rimanere 65{-}35.\newline%
%
\newline%
%
Wa e Wl sono le qunatità percentuali delle due fasi, rispettivamente solida e liquida\newline%
%
\newline%
%
\begin{alignat}{2}%
Wa + Wl = 1
%
\end{alignat}%
La somma delle percentuali devono date 100\%\newline%
%
\newline%
%
\begin{alignat}{2}%
Wa*Ca + Wl*Cl = C0
%
\end{alignat}%
C0 è la concentrazione iniziale ad esempio nello stato monofasico solido\newline%
%
\newline%
%
Le formule seguenti ci permettono quindi di calcolare la quantità delle fasi\newline%
%
Viene chiamata regola della leva\newline%
%
\newline%
%
\begin{alignat}{2}%
Wl = (Ca - C0) / (Ca - Cl)
%
\end{alignat}%
\begin{alignat}{2}%
Wa = (C0 - Cl) / (Ca - Cl)
%
\end{alignat}%
\newline%

%
\subsection{ Rame Nichel Solidificazione }%
\label{subsec:RameNichelSolidificazione}%
Lega Rame{-}Nichel 65{-}35\newline%
%
\newline%
%
Punto b appena sotto alla linea di liquidus ovvero è appena cominciato il processo di Solidificazione\newline%
%
Presenza di due fasi: composizione della fase liquida è circa 65{-}35, la composizione del solido sarà 46\% Nichel.\newline%
%
La temperaatura di fusione del Nichel è più alta di quella del rame, di conseguenza abbassando la temperatura si solidificherà per primo.\newline%
%
\newline%
%
Punto c al di sotto del punto b, si trova quandi ad una temperatura più bassa.\newline%
%
In questo punto anche il rame avrà cominciato a sciogliersi, quindi la composizione sarà nella fase liquida 32\% Nichel e nella fase solida 43\% Nichel.\newline%
%
\newline%
%
Punto d appena prima della solidificazione completa.\newline%
%
In questo punto ci sarà pochissima quantità di fase liquida e questa avrà una concentrazione molto alta di rame,\newline%
%
infatti nella fase liquida la composizione sarà 24\% Nichel.\newline%
%
La fase solida avrà una composizione molto simile alla composizione originale ovvero 35\% Nichel.\newline%
%
\newline%

%
\subsection{ Andamento della Solidificazione }%
\label{subsec:AndamentodellaSolidificazione}%
diagramma che rappresenta l'andamento della frazione volumetrica di materiale trasformata in funzione del logaritmo del tempo\newline%
%
condotta a temperatura costante.\newline%
%
\newline%
%
E' unandamento a sigmoide infatti ha una froma a "S" schiacciata\newline%
%
\newline%
%
Tempo di nucleazione, tempo necessario perché la nucleazione inizi, tempo in cui si creano gli embrioni, termina quando dagli embriovi si creano i nuclei.\newline%
%
Da questo tempo comincia la crescita che continua fino al 100\%\newline%
%
Il punto di completa trasformazione è complicato da calcolare, infatti la trasformazione ha un andamento asintotico.\newline%
%
E più semplice determinare il tempo al 50\% della trasformazione.\newline%
%
\newline%

%
\subsection{ Riscristallizazione }%
\label{subsec:Riscristallizazione}%
I cristalli si formano nella Solidificazione, ma in funzione della temperatura è possibile che durante los stato solido si formino altri cristalli.\newline%
%
Ad esempio in seguito a deformazioni di creano dislocazioni e il processo di ricristallizzazione permette al metallo di\newline%
%
rimuovere queste deformazione ricristalizzando il metallo.\newline%
%
\newline%
%
Questo processo dipende dalla temperatura, infatti a 135 °C comincia dopo un paio di secondi e termina dopo circa 30,\newline%
%
a 34°C comincia dopo un decina di minuti e termina in una decina di ore.\newline%

%
\end{document}